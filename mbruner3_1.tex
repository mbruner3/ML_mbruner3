% Options for packages loaded elsewhere
\PassOptionsToPackage{unicode}{hyperref}
\PassOptionsToPackage{hyphens}{url}
%
\documentclass[
]{article}
\usepackage{lmodern}
\usepackage{amssymb,amsmath}
\usepackage{ifxetex,ifluatex}
\ifnum 0\ifxetex 1\fi\ifluatex 1\fi=0 % if pdftex
  \usepackage[T1]{fontenc}
  \usepackage[utf8]{inputenc}
  \usepackage{textcomp} % provide euro and other symbols
\else % if luatex or xetex
  \usepackage{unicode-math}
  \defaultfontfeatures{Scale=MatchLowercase}
  \defaultfontfeatures[\rmfamily]{Ligatures=TeX,Scale=1}
\fi
% Use upquote if available, for straight quotes in verbatim environments
\IfFileExists{upquote.sty}{\usepackage{upquote}}{}
\IfFileExists{microtype.sty}{% use microtype if available
  \usepackage[]{microtype}
  \UseMicrotypeSet[protrusion]{basicmath} % disable protrusion for tt fonts
}{}
\makeatletter
\@ifundefined{KOMAClassName}{% if non-KOMA class
  \IfFileExists{parskip.sty}{%
    \usepackage{parskip}
  }{% else
    \setlength{\parindent}{0pt}
    \setlength{\parskip}{6pt plus 2pt minus 1pt}}
}{% if KOMA class
  \KOMAoptions{parskip=half}}
\makeatother
\usepackage{xcolor}
\IfFileExists{xurl.sty}{\usepackage{xurl}}{} % add URL line breaks if available
\IfFileExists{bookmark.sty}{\usepackage{bookmark}}{\usepackage{hyperref}}
\hypersetup{
  pdftitle={mbruner3\_1.r},
  pdfauthor={Mark Bruner},
  hidelinks,
  pdfcreator={LaTeX via pandoc}}
\urlstyle{same} % disable monospaced font for URLs
\usepackage[margin=1in]{geometry}
\usepackage{color}
\usepackage{fancyvrb}
\newcommand{\VerbBar}{|}
\newcommand{\VERB}{\Verb[commandchars=\\\{\}]}
\DefineVerbatimEnvironment{Highlighting}{Verbatim}{commandchars=\\\{\}}
% Add ',fontsize=\small' for more characters per line
\usepackage{framed}
\definecolor{shadecolor}{RGB}{248,248,248}
\newenvironment{Shaded}{\begin{snugshade}}{\end{snugshade}}
\newcommand{\AlertTok}[1]{\textcolor[rgb]{0.94,0.16,0.16}{#1}}
\newcommand{\AnnotationTok}[1]{\textcolor[rgb]{0.56,0.35,0.01}{\textbf{\textit{#1}}}}
\newcommand{\AttributeTok}[1]{\textcolor[rgb]{0.77,0.63,0.00}{#1}}
\newcommand{\BaseNTok}[1]{\textcolor[rgb]{0.00,0.00,0.81}{#1}}
\newcommand{\BuiltInTok}[1]{#1}
\newcommand{\CharTok}[1]{\textcolor[rgb]{0.31,0.60,0.02}{#1}}
\newcommand{\CommentTok}[1]{\textcolor[rgb]{0.56,0.35,0.01}{\textit{#1}}}
\newcommand{\CommentVarTok}[1]{\textcolor[rgb]{0.56,0.35,0.01}{\textbf{\textit{#1}}}}
\newcommand{\ConstantTok}[1]{\textcolor[rgb]{0.00,0.00,0.00}{#1}}
\newcommand{\ControlFlowTok}[1]{\textcolor[rgb]{0.13,0.29,0.53}{\textbf{#1}}}
\newcommand{\DataTypeTok}[1]{\textcolor[rgb]{0.13,0.29,0.53}{#1}}
\newcommand{\DecValTok}[1]{\textcolor[rgb]{0.00,0.00,0.81}{#1}}
\newcommand{\DocumentationTok}[1]{\textcolor[rgb]{0.56,0.35,0.01}{\textbf{\textit{#1}}}}
\newcommand{\ErrorTok}[1]{\textcolor[rgb]{0.64,0.00,0.00}{\textbf{#1}}}
\newcommand{\ExtensionTok}[1]{#1}
\newcommand{\FloatTok}[1]{\textcolor[rgb]{0.00,0.00,0.81}{#1}}
\newcommand{\FunctionTok}[1]{\textcolor[rgb]{0.00,0.00,0.00}{#1}}
\newcommand{\ImportTok}[1]{#1}
\newcommand{\InformationTok}[1]{\textcolor[rgb]{0.56,0.35,0.01}{\textbf{\textit{#1}}}}
\newcommand{\KeywordTok}[1]{\textcolor[rgb]{0.13,0.29,0.53}{\textbf{#1}}}
\newcommand{\NormalTok}[1]{#1}
\newcommand{\OperatorTok}[1]{\textcolor[rgb]{0.81,0.36,0.00}{\textbf{#1}}}
\newcommand{\OtherTok}[1]{\textcolor[rgb]{0.56,0.35,0.01}{#1}}
\newcommand{\PreprocessorTok}[1]{\textcolor[rgb]{0.56,0.35,0.01}{\textit{#1}}}
\newcommand{\RegionMarkerTok}[1]{#1}
\newcommand{\SpecialCharTok}[1]{\textcolor[rgb]{0.00,0.00,0.00}{#1}}
\newcommand{\SpecialStringTok}[1]{\textcolor[rgb]{0.31,0.60,0.02}{#1}}
\newcommand{\StringTok}[1]{\textcolor[rgb]{0.31,0.60,0.02}{#1}}
\newcommand{\VariableTok}[1]{\textcolor[rgb]{0.00,0.00,0.00}{#1}}
\newcommand{\VerbatimStringTok}[1]{\textcolor[rgb]{0.31,0.60,0.02}{#1}}
\newcommand{\WarningTok}[1]{\textcolor[rgb]{0.56,0.35,0.01}{\textbf{\textit{#1}}}}
\usepackage{graphicx,grffile}
\makeatletter
\def\maxwidth{\ifdim\Gin@nat@width>\linewidth\linewidth\else\Gin@nat@width\fi}
\def\maxheight{\ifdim\Gin@nat@height>\textheight\textheight\else\Gin@nat@height\fi}
\makeatother
% Scale images if necessary, so that they will not overflow the page
% margins by default, and it is still possible to overwrite the defaults
% using explicit options in \includegraphics[width, height, ...]{}
\setkeys{Gin}{width=\maxwidth,height=\maxheight,keepaspectratio}
% Set default figure placement to htbp
\makeatletter
\def\fps@figure{htbp}
\makeatother
\setlength{\emergencystretch}{3em} % prevent overfull lines
\providecommand{\tightlist}{%
  \setlength{\itemsep}{0pt}\setlength{\parskip}{0pt}}
\setcounter{secnumdepth}{-\maxdimen} % remove section numbering

\title{mbruner3\_1.r}
\author{Mark Bruner}
\date{9/11/2020}

\begin{document}
\maketitle

\hypertarget{reference}{%
\subsection{Reference}\label{reference}}

Data is taken from
\url{http://faculty.marshall.usc.edu/gareth-james/ISL/College.csv}

\hypertarget{start}{%
\subsection{Start}\label{start}}

Loaded in the college.csv and assigned it to the variable ``college''.
Renamed the first column in the data set.

\begin{Shaded}
\begin{Highlighting}[]
\KeywordTok{library}\NormalTok{(readr)}
\KeywordTok{library}\NormalTok{(dplyr)}
\NormalTok{college <-}\StringTok{ }\KeywordTok{read_csv}\NormalTok{(}\StringTok{"college.csv"}\NormalTok{, }\DataTypeTok{col_types =} \StringTok{"cciiiiiiiiiiiiiiiii"}\NormalTok{)}
\NormalTok{college <-}\StringTok{ }\KeywordTok{rename}\NormalTok{(college, }\StringTok{"College Name"}\NormalTok{ =}\StringTok{ "X1"}\NormalTok{)}
\NormalTok{college }
\end{Highlighting}
\end{Shaded}

\begin{verbatim}
## # A tibble: 777 x 19
##    `College Name` Private  Apps Accept Enroll Top10perc Top25perc F.Undergrad
##    <chr>          <chr>   <int>  <int>  <int>     <int>     <int>       <int>
##  1 Abilene Chris~ Yes      1660   1232    721        23        52        2885
##  2 Adelphi Unive~ Yes      2186   1924    512        16        29        2683
##  3 Adrian College Yes      1428   1097    336        22        50        1036
##  4 Agnes Scott C~ Yes       417    349    137        60        89         510
##  5 Alaska Pacifi~ Yes       193    146     55        16        44         249
##  6 Albertson Col~ Yes       587    479    158        38        62         678
##  7 Albertus Magn~ Yes       353    340    103        17        45         416
##  8 Albion College Yes      1899   1720    489        37        68        1594
##  9 Albright Coll~ Yes      1038    839    227        30        63         973
## 10 Alderson-Broa~ Yes       582    498    172        21        44         799
## # ... with 767 more rows, and 11 more variables: P.Undergrad <int>,
## #   Outstate <int>, Room.Board <int>, Books <int>, Personal <int>, PhD <int>,
## #   Terminal <int>, S.F.Ratio <int>, perc.alumni <int>, Expend <int>,
## #   Grad.Rate <int>
\end{verbatim}

Loaded ``summarytools'' package.

\begin{Shaded}
\begin{Highlighting}[]
\KeywordTok{library}\NormalTok{(summarytools)}
\end{Highlighting}
\end{Shaded}

\hypertarget{quantitative-descriptive-statistics}{%
\section{Quantitative Descriptive
Statistics}\label{quantitative-descriptive-statistics}}

Made the ``Apps'' column in college data set a variable and ran summary
of desc. statistcs.

\begin{Shaded}
\begin{Highlighting}[]
\NormalTok{apps <-}\StringTok{ }\NormalTok{college}\OperatorTok{$}\NormalTok{Apps}
\KeywordTok{descr}\NormalTok{(}\DataTypeTok{x =}\NormalTok{ apps)}
\end{Highlighting}
\end{Shaded}

\begin{verbatim}
## Descriptive Statistics  
## apps  
## N: 777  
## 
##                         apps
## ----------------- ----------
##              Mean    3001.64
##           Std.Dev    3870.20
##               Min      81.00
##                Q1     776.00
##            Median    1558.00
##                Q3    3624.00
##               Max   48094.00
##               MAD    1463.33
##               IQR    2848.00
##                CV       1.29
##          Skewness       3.71
##       SE.Skewness       0.09
##          Kurtosis      26.52
##           N.Valid     777.00
##         Pct.Valid     100.00
\end{verbatim}

\#\#Qualitative Descriptive Statistics \& Transformation of Data Used
the ``Private'' column in the College data set and converted it to a DF.
After that I added a column to the DF to represent the percentage.

\begin{Shaded}
\begin{Highlighting}[]
\KeywordTok{library}\NormalTok{(tidyverse)}
\NormalTok{private <-}\StringTok{ }\KeywordTok{table}\NormalTok{(college}\OperatorTok{$}\NormalTok{Private)}
\NormalTok{private_df <-}\StringTok{ }\KeywordTok{as.data.frame}\NormalTok{(private)}
\NormalTok{private_df <-}\StringTok{ }\KeywordTok{rename}\NormalTok{(private_df, }\StringTok{"Private"}\NormalTok{ =}\StringTok{ "Var1"}\NormalTok{)}
\NormalTok{private_percent <-}\StringTok{ }\NormalTok{private_df}\OperatorTok{$}\NormalTok{Freq}\OperatorTok{/}\KeywordTok{sum}\NormalTok{(private_df}\OperatorTok{$}\NormalTok{Freq)}
\NormalTok{private_df_percent <-}\StringTok{ }\KeywordTok{add_column}\NormalTok{(private_df, private_percent)}
\NormalTok{private_df_percent <-}\StringTok{ }\KeywordTok{rename}\NormalTok{(private_df_percent, }\StringTok{"Percent"}\NormalTok{ =}\StringTok{ "private_percent"}\NormalTok{)}
\NormalTok{private_df_percent}
\end{Highlighting}
\end{Shaded}

\begin{verbatim}
##   Private Freq   Percent
## 1      No  212 0.2728443
## 2     Yes  565 0.7271557
\end{verbatim}

\hypertarget{scatterplot-multi-variable}{%
\subsection{Scatterplot:
Multi-variable}\label{scatterplot-multi-variable}}

Created a scatterplot of the two variable Applications and Acceptance in
the College Data Set.

\begin{Shaded}
\begin{Highlighting}[]
\KeywordTok{library}\NormalTok{(ggplot2)}
\NormalTok{Applications =}\StringTok{ }\NormalTok{college}\OperatorTok{$}\NormalTok{Apps}
\NormalTok{Acceptance =}\StringTok{ }\NormalTok{college}\OperatorTok{$}\NormalTok{Accept}
\NormalTok{Private =}\StringTok{ }\NormalTok{college}\OperatorTok{$}\NormalTok{Private}
\KeywordTok{ggplot}\NormalTok{(}\DataTypeTok{data =}\NormalTok{ college, }\KeywordTok{aes}\NormalTok{(}\DataTypeTok{x =}\NormalTok{ Applications, }\DataTypeTok{y =}\NormalTok{ Acceptance, }\DataTypeTok{col =}\NormalTok{ Private, }\DataTypeTok{shape =}\NormalTok{ Private)) }\OperatorTok{+}\StringTok{ }\KeywordTok{geom_point}\NormalTok{()}
\end{Highlighting}
\end{Shaded}

\includegraphics{mbruner3_1_files/figure-latex/unnamed-chunk-5-1.pdf}

\hypertarget{single-variable-visualization}{%
\section{Single Variable
Visualization}\label{single-variable-visualization}}

ggplot(data = college, aes(x = Private)) + geom\_histogram(stat =
``Count'', fill = ``light blue'', col = ``skyblue'')

\begin{Shaded}
\begin{Highlighting}[]
\KeywordTok{ggplot}\NormalTok{(}\DataTypeTok{data =}\NormalTok{ college, }\KeywordTok{aes}\NormalTok{(}\DataTypeTok{x =}\NormalTok{ Private)) }\OperatorTok{+}\StringTok{ }\KeywordTok{geom_histogram}\NormalTok{(}\DataTypeTok{stat =} \StringTok{"Count"}\NormalTok{, }\DataTypeTok{fill =} \StringTok{"light blue"}\NormalTok{, }\DataTypeTok{col =} \StringTok{"skyblue"}\NormalTok{)}
\end{Highlighting}
\end{Shaded}

\includegraphics{mbruner3_1_files/figure-latex/unnamed-chunk-6-1.pdf}

\end{document}
